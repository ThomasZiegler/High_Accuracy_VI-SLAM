\chapter{Conclusion and Outlook} 
\label{ch:conclusion}


\section{Conclusion}
In this semester project, an existing open source monocular keyframe based 
\ac{VIO} pipeline was adapted to achieve accurate pose estimation, while having 
as little computational demand as possible. Our approach is based on VINS-Mono 
\citep{Qin2017VINS}, which was chosen after a comparison of the publicly 
available monocular \ac{VIO} pipelines. \\

An extensive evaluation of the default VINS-Mono pipeline parameters was 
performed to determine their influence on the accuracy and the per-frame 
optimization time. Based on these evaluation an approach was proposed, which 
ignores any features in the \ac{BA} optimization, if the corresponding landmark 
has been observed in one of the oldest four keyframes. This means, the oldest 
four keyframes do not provide any feature constraints. This reduces the number 
of freely adjustable parameters in the optimization. In addition, only features 
that have been observed in at least six keyframes in the window are marginalized 
out. Finally, in order to have a save margin in the optimization time, the 
number of tracked features was reduced to 120. \\

The proposed adaption were then extensively tested on all eleven sequences of 
the EuRoC dataset. The results show that our approach is less accurate than the 
default implementation of VINS-Mono, as expected, but is very consistent over 
the three 
tested platforms. This includes two computational limited single board 
computers and a much stronger Laptop. The similar results proof the robustness 
of our algorithm. Furthermore, the test shows that even the outliers in the 
per-frame optimization time do not exceed the 100 millisecond, ensuring 
real-time applicability. The RMSE errors show that our approach is most suitable 
for slower movements. Whereas with faster movements, the reduced number of 
features in the optimization becomes noticeable. This suggests, that as one 
might expect, there is no free lunch in visual state estimation. \\

We also tested the proposed \ac{VIO} pipeline with a \ac{MSF} framework on a 
\ac{UAV}, which succeeded to hover. However, there was noticeable incertitude 
in the movement. Due to time limits, this effects could not be further 
investigated. However we assume, that most of this incertitude can be 
eliminated by properly tuning the \ac{MSF} parameters and performing a more 
accurate calibration.\\

We want to note that in our first approach we only ignored the feature 
constraints in the oldest four keyframes of the sliding window, as described in 
\autoref{sec:dontOptimizeOldest}. Which is equivalent to perform only a pose 
graph optimization for the oldest keyframes. However, we had to reduce the 
number of tracked features down to 80 to achieve real-time performance in all 
of the tested sequences described in \autoref{subsec:experiment1}. Otherwise, 
there were always a few outliers in the total per-frame optimization time above 
100 milliseconds. Reducing the tracked feature that much reduced also the 
accuracy significantly. This led us to the approach of ignoring all feature
constraints in the \ac{BA} optimization if a feature has been observed in one 
of the oldest keyframes. 

\section{Outlook}
The evaluation is done on the different summands of the \ac{BA} 
optimization, but this does 
not cover all of the adjustable parameters. Some of the parameters that have 
not been analyzed in this work are the following:
\begin{itemize}
 \item Minimum distance between two features.
 \item Keyframe selection threshold (parallax and minimum detected features).
 \item Solver types in Google's \textit{ceres solver}\citep{ceres-solver}.
\end{itemize}
These parameters might have a significant influence on the accuracy or the 
per-frame optimization time. \\

The \ac{ROS} implementation of the default VINS-Mono and our adaption contains 
buffers. However, when using as real-time \ac{VIO} pipeline, it would be better 
to drop frames if the optimization can't catch up with the feature tracker. 
Otherwise, the \ac{BA} optimization may be performed on old data. Hence, one 
could create a ``live'' version of our \ac{VIO} pipeline, which does not use 
buffers. \\

Ignoring features seen in the oldest keyframes is a quite radical approach for 
reducing the number of feature constraints. A more sophisticated heuristic for 
reducing the number of feature constraints may lead to an even better time 
accuracy trade off.

