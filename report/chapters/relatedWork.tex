\chapter{Related Work}
\label{sec:relatedWork}
%\chapter{Einleitung}
%\label{sec:einleitung}

There exists extensive research on vision-based odometry/\ac{SLAM} systems. 
However, in 
this chapter we skip a full discussion on the available literature and focus 
only on work that this semester project relies on. In particular, a benchmark 
comparison of different \ac{VIO} systems and a brief discussion on the publicly
available \ac{VIO} and VI-SLAM systems. The chapter ends with a short 
comparison 
of the suitability of these systems regarding the goals of this work. 


\section{A Benchmark Comparison}
In \citep{Delmerico2018Benchmark}, an extensive comparison of publicly 
available \ac{VIO} pipelines (MSCKF, OKVIS, ROVIO, VINS-Mono, SVO+MSF, and 
SVO+GTSAM) is performed with respect to the per-frame processing time, CPU and 
memory load. The algorithms are evaluated on all sequences of the EuRoC 
datasets while processed on different single board computer systems. The 
results are presented as a benchmark. 



\section{MSCKF}
The Multi-State constraint Kalman Filter (MSCKF) \citep{Mourikis2007MSCKF} is a 
popular \ac{EKF} based \ac{VIO} system. It maintains several previous camera 
poses in 
the state vector and uses geometric constraint between all the poses that 
observe a particular feature as update. There exist a publicly available 
implementations of this 
algorithm\footnote{https://github.com/daniilidis-group/msckf\_mono}.


\section{OKVIS}
Open Keyframe-based Visual-Inertial SLAM (OKVIS) \citep{Leutenegger2015} is a 
state-of-the-art \ac{SLAM} system that works with monocular and stereo cameras. 
However, it should be noted that it is not optimized for monocular \ac{VIO} and 
its performance is superior in the stereo setup. It utilizes non-linear 
\ac{BA} optimization with a sliding window of keyframe poses. Keyframes older 
than the 
sliding window are marginalized out in order to retrieve the constraints. The 
source code of a \ac{ROS} implementation has been made publicly 
available\footnote{https://github.com/ethz-asl/okvis\_ros}.


\section{ROVIO}
Robust Visual Inertial Odometry (ROVIO) \citep{Bloesch2015ROVIO} is a 
visual-inertial estimator based on an \ac{EKF}. This\ac{VIO} system uses 
the photometric error of tracked image patches to find the optimal state in the 
update step. The framework is specifically designed for monocular systems and 
does not require any special initialization procedure. The source code of a 
\ac{ROS} implementation has been made publicly 
available\footnote{https://github.com/ethz-asl/rovio}.  


\section{SVO+GTSAM}
Semi-Direct Monocular Visual Odometry (SVO) \citep{Forster2016SVO} is a 
computationally lightweight \ac{VO} algorithm which employs sparse image 
alignment to estimate motion. It tracks features by minimizing the photometric 
error of image patches around the features between subsequent frames. The 
tracking is tightly coupled with an online factor graph optimization based on 
iSAM2 \citep{Kaess2014iSAM2} for the state estimation of selected keyframes. The 
\ac{IMU} measurements between two consecutive keyframes are pre-integrated as 
described in \citep{Forster2017Manifold} in order to be used in the 
optimization. Both components SVO\footnote{http://rpg.ifi.uzh.ch/svo2.html}, and 
iSAM2 implemented in the GTSAM 4.0 optimization 
toolbox\footnote{https://bitbucket.org/gtborg/gtsam/} \citep{gtsam}, are 
publicly available. However, the integration of these two systems is not 
published and SVO is only provided as binaries.

\section{VINS-Mono}
VINS-Mono, presented in \citep{Qin2017VINS} is a complete VI-\ac{SLAM} system. 
It uses 
non-linear \ac{BA} optimization over a sliding window for the pose estimation. 
Robust corner 
features are tracked over consecutive frames. Similar to OKVIS, the 
optimization 
is performed over keyframes, which are marginalized out when they leave the 
window. Similar to SVO, the \ac{IMU} measurements between consecutive keyframes 
are pre-integrated to reduce the computational demand. VINS-Mono also provides 
a tightly integrated relocalization. Together with the 4 \ac{DoF} pose 
graph optimization it builds the back-end of the \ac{SLAM} system. The source 
code is available as a ROS compatible 
implementation\footnote{https://github.com/HKUST-Aerial-Robotics/VINS-Mono}.


\section{Summary}
As stated in the introduction, in \autoref{sec:introduction}, the goal of this 
work is to implement a low demanding monocular keyframe based \ac{VIO} system. 
From the above mentioned systems MSCKF and ROVIO are both filter based and do 
not have a keyframe selection. SVO+GTSAM requires the source code to implement 
the tightly coupled system. The source code of SVO is only available in the 
initial version, which does not support front-looking cameras. The remaining 
systems are OKVIS and VINS-Mono. OKVIS is optimized for stereo cameras and has 
to re-integrate the \ac{IMU} measurements in every optimization step, leading 
to computational overhead. VINS-Mono, on the other hand, is designed for a 
monocular setup and does support pre-integration of the \ac{IMU} measurements in 
order to avoid repeated \ac{IMU} re-integration. We therefore decided to use 
VINS-Mono as basis for this work. 
