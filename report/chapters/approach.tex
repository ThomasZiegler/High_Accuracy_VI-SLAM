\chapter{Approach} 
\label{ch:approach}

As described in the evaluation, the overall optimization time consists of the 
time to solve the \ac{BA} optimization and the time needed for the 
marginalization. The runtime for the \ac{BA} optimization is limited, but the 
accuracy can be affected depending on the optimization settings. The 
marginalization time, on the other hand, depends on the number of features to be 
marginalized. \\

To achieve real-time performance for our \ac{VIO} pipeline, the total 
optimization time has to be below 100 milliseconds as our system runs with a 
refresh rate of 10Hz. Based on the evaluation in \autoref{ch:evaluation}, we 
propose an approach which fulfills this real-time requirement. \\

Our approach reduces the number of tracked features 
$n_{\text{features\_tracked}}$ and combines the optimization- and 
marginalization-settings which lead to the best trade off between time and 
accuracy. In the optimization we ignore landmarks which were observed in at 
least one of a certain number of old keyframes $n_{\text{oldest\_keyframes}}$. 
In the marginalization we do not marginalize landmarks which are observed in 
less than a certain number of (key-)frames $n_{\text{observed\_keyframes}}$. \\

In summary, the proposed changes are: 
\begin{itemize}
  \item Reduce the number of tracked features from 150 down to 
$n_{\text{features\_tracked}}$.
  \item Ignoring feature constraints in the \ac{BA} optimization, if the 
corresponding landmark has been observed in at least one of the 
$n_{\text{oldest\_keyframes}}$ oldest keyframes.
  \item Only marginalize out features that have been observed in more than 
$n_{\text{observed\_keyframes}}$ (key-)frames. 
\end{itemize}

